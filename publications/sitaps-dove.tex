\documentclass{article}
\usepackage{listings}

\begin{document}
\title{Dove and Ivory: Verifying One DSL with Another}
\author{Tom Hawkins, Howard Reubenstein, Greg Eakman, BAE Systems Inc.}
\date{\today}
\maketitle

\begin{abstract}
Dove (a \underline{D}SL \underline{O}perational \underline{V}erification \underline{E}nvironment)
is a language to capture verification conditions (VCs) for the 
Ivory DSL for verifying user and compiler generated assertions
and procedure contracts.  VCs from an Ivory program are first
translated into Dove then into ACL2 for formal analysis.
Verified assertions and contracts are then optimized out of
the Ivory program for improved runtime performance.
This paper provides an overview of the Dove language,
the VC generation strategy, the translation from Ivory to Dove,
and from Dove to ACL2.  Verification results are highlighted
and future extensions discussed.
\end{abstract}

% http://www.cs.utexas.edu/users/moore/acl2/workshop-2015/index.html

\section{Problem Definition}

Developed under DARPA HACMS, Ivory is a DSL in Haskell for 
embedded programming.  Ivory is similar to C but
enforces memory safety though its type system.
To capture design intent, Ivory provides
assertions, assumptions, and procedure contracts.
In addition, the Ivory compiler generates
assertions to guard a program against a host of runtime violations including
floating point exceptions, numerical overflows, and unbounded loops.

Verification of these assertions is crucial for two reasons.  First and foremost,
assertions are still runtime checks, and failure of such are equivalent 
to uncaught exceptions (think Ariane 5).  This importance cannot
be underscored in HACMS, since an Ivory autopilot will be flying a real helicopter
with an onboard safety pilot.  Second, runtime checks have
runtime overhead: if these checks and their associated logic can be safely
removed, memory consumption and execution time are reduced; important 
for embedded system, which often run under tight resource constrains.

\subsection{Interprocedural Analysis with Dove}
TODO
\begin{itemize}
  \item Overview of interprocedural (used for scalability): Assume requires and sub-ensures, check sub-requires, ensures, and asserts.
  \item Describe the iterative VC generation / VC verification process; accumulating lemmas as it goes.
  \item Punt on recursion: all procedure requires left in place.
  \item Punt on loops.
  \item Punt on floating point.
  \item Punt on stack across procedure calls.
\end{itemize}

\section{Dove Language}
The Dove language does not have a concrete syntax.  Figure \ref{doveSyntax} lists the various expressions of Dove.


\begin{figure}
  \caption{Dove Syntax}
  \label{doveSyntax}
  \begin{lstlisting}
data Expr
  -- Variable reference.
  = Var           String

  -- Literals.
  | Unit
  | Bool          Bool
  | Integer       Integer

  -- Introduce a variable bound to an expression.
  | Let           String Expr Expr

  -- Introduce a universally quantified variable.
  | ForAll        String Expr

  -- Conditional, uniary, and binary operations.
  | If            Expr Expr Expr
  | UniOp         UniOp Expr
  | BinOp         BinOp Expr Expr

  -- Array construction.
  | Array         [Expr]

  -- Array appending.
  | ArrayAppend   Expr Expr

  -- Array projecting (array, index).
  | ArrayProject  Expr Expr

  -- Updating an array with an index and a new value (index, value, oldArray).
  | ArrayUpdate   Expr Expr Expr

  -- Record construction.
  | Record        [(String, Expr)]

  -- Record projection.
  | RecordProject Expr String

  -- Create a new record by overlaying one record over another.
  | RecordOverlay Expr Expr

  -- Comments for annotation.
  | Comment       String Expr
  \end{lstlisting}
\end{figure}

\begin{figure}
  \caption{Dove Uniary and Binary Operators}
  \label{doveOperators}
  \begin{lstlisting}
data UniOp
  = Not      -- Boolean negation.
  | Length   -- Length of an array.
  | Negate   -- Integer negation.
  | Abs      -- Absolute value.
  | IsArray  -- Array type predicate.
  | IsInt    -- Integer type predicate.

data BinOp
  = And      -- Boolean AND.
  | Or       -- Boolean OR.
  | Implies  -- Boolean implication.
  | Eq       -- Equal comparison.
  | Lt       -- Less than comparison.
  | Gt       -- Greater than comparison.
  | Add      -- Integer addition.
  | Sub      -- Integer subtraction.
  | Mul      -- Integer multiplication.
  | Mod      -- Integer modulo.
  \end{lstlisting}
\end{figure}

TODO
(Maybe just for the SITAPS report not the paper?)
\begin{itemize}
  \item Basic types: booleans, integers
  \item Aggregate types: arrays, records
  \item Expressions:
  \begin{itemize}
    \item Literals: true, false, 22
    \item Conditional: if ... then ... else
    \item Basic operations: logic, arithmetic, comparison
    \item Array operations: append, project, update
    \item Record operations: project, overlay
    \item Variable introduction:
    \begin{itemize}
      \item Let a = 22 in ...
      \item Forall a in ...
    \end{itemize}
  \end{itemize}
\end{itemize}

\section{Translating Ivory to Dove}
TODO
\begin{itemize}
  \item Both Ivory and Dove are DSLs embedded in Haskell.  Makes for a clean AST to AST translation.
  \item Detail how Ivory AST elements are encoded into Dove.
  \item Comments are good: Ivory AST elements are heavily commented and hierarchically group during translation.
\end{itemize}

\section{Translating Dove to ACL2}
TODO
\subsection{Dove Optimization}
\subsection{ACL2 Translation}

\section{Practical Verification}
TODO
\begin{itemize}
  \item Attempted to make these industrial ready tools, not just a research project.
  \item Controlled verbosity.  Output progress, failures, VC, optimized VC, ACL2, ACL2 result.
\end{itemize}

\section{Results}
TODO
\subsection{Results from Testsuite}
\subsection{Results from Quadcopter Software}
Testing over (can we get Galois to do anything with this?)

\section{Related and Future Work}
(For conference paper the related work section will need to be 
pretty strong -- maybe we can get some additional eyes on this -- 
of course it may turn out that this work is too similar -- interesting
but still too similar -- to some already published approach)

Similar systems:
\begin{itemize}
  \item Frama-C (Why3)
  \item Spark Ada
\end{itemize}

\subsection{Future Work}
\begin{itemize}
  \item Address limitations: no floating point, no stack across calls, no loops, bitwise operations.
  \item Incorporate other backends (SMT, WHY3).
  \item Target other languages: pointer operations (Cish); Breeze type, contract, and maybe IFC checking; HDLs.
\end{itemize}

\section{Conclusions}

\end{document}

